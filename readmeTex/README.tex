\documentclass[letterpaper,11pt]{article}

\usepackage[utf8]{inputenc}
\usepackage[spanish,mexico]{babel}
\usepackage{graphicx}
\usepackage{amsmath}
\usepackage{amsthm}
\usepackage{svg}
\usepackage{amsfonts}
\usepackage{subcaption}
\usepackage[hmargin=1in, vmargin=1in]{geometry}
\usepackage{fancyhdr}
\pagestyle{fancy}
\usepackage{tasks}
\lhead{\ExiCarrera}
\chead{\ExiMateria}
\rhead{\ExiParcial}
\cfoot{\ExiEscuela}
\renewcommand{\headrulewidth}{0.4pt}
\renewcommand{\footrulewidth}{0.4pt}

\providecommand{\abs}[1]{\lvert#1\rvert}
\providecommand{\norm}[1]{\lVert#1\rVert}

\newcommand{\informacion}[1]{
\begin{center}
\fbox{\fbox{\parbox{\textwidth}{{\footnotesize#1}}}}
\end{center}
\vspace{5mm}}

\begin{document}
\setlength{\unitlength}{1cm}
\thispagestyle{empty}
\begin{picture}(18,4)
\put(-0.5,1.2){\includegraphics[scale=.25]{unam1.png}}
\put(13.5,1){\includegraphics[scale=.35]{fciencias1.png}}
\end{picture}

\begin{center}
\vspace{-134pt}
\textbf{\large Estructuras de Datos}\\[0.2cm]
\textbf{ Semestre 2020-2}\\[0.2cm]
Prof. Alejandro Hernández Mora\\[0.2cm]
Ayud. Pablo Camacho González  \\ [0.2cm]
Ayud. Lab. Luis Manuel Martínez Dámaso   \\ [0.2cm]
\rule{17cm}{0.3mm}\\
\textbf{Tarea 3}\\
\huge\textbf{Diccionario}\\[0.1cm]
\normalsize Kevin Ariel Merino Peña\footnote{317031326}\\
Armando Abraham Aquino Chapa\footnote{317058163}\\
\end{center}
\vspace{-10pt}
\rule{17cm}{0.3mm}
\begin{flushright}
\vspace{-3pt}
\end{flushright}

\noindent Durante la ejecución del programa siga las siguientes instrucciones

\section*{Instrucciones}

\begin{enumerate}

\item Para ejecutar el programa posicionarse en el directorio que contiene \textbf{Main.java} y compilar con todas las clases necesarias.

\item El programa ofrece dos opciones, en la primera: se puede consultar si una palabra es correcta, se revisa si una letra está mal escrtia o si le falta algún acento, suguiere un conjunto de opciones que podrían acercarse a una palabra en el diccionario en caso de que la entrada no exista en el diccionario.

\item En la segunda opción se permite: salir

\end{enumerate}
\section*{Aclaraciones}
	Aquí se especifica que el diccionario funcionará mejor si sólo se tratra de un error en mayusculas, acentos o la palabra
	proveida está mal escrita por una letra. Debido a que no es la finalidad del curso implementar las mejores sugerencias, se prefirió
	ilustrar el uso de la estructura árbol AVL


\end{document}
